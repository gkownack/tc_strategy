\documentclass{article}

\usepackage{graphicx}

\begin{document}

% Letter of transmittal
\thispagestyle{empty}
\noindent
CMU SMC 4349\\
5000 Forbes Avenue\\
Pittsburgh, PA, 15289

\vspace{2em}

\noindent
May 7, 2014

\vspace{2em}

\noindent
Mr. Thomas M. Keating\\
Assistant Teaching Professor\\
School of Computer Science\\
Pittsburgh, PA, 15289

\vspace{2em}

\noindent
Dear Mr. Keating,

\vspace{2em}

\noindent
Included with this letter is the final report for our team's project,
    ``Iron Legacy.''
     This report covers our team's approach to the project and the project
     results.

\vspace{2em}

\noindent
This report details the problem that our project addresses, 
    as well as our solution.
    The report also lists our projects goals
    and describes the technologies we used and considered.
    Finally, the report details the results and includes an evaluation of our
    team's work, the lessons we learned, and advice for future groups.

\vspace{2em}

\noindent
If you have questions about our project and wish to contact us,
    please send an email to wtwood@andrew.cmu.edu.

\vspace{2em}

\noindent
Sincerely,

\vspace{4em}

\noindent
Billy Wood

\vfill

\noindent
enclosure: paper entitled ``Iron Legacy''

\clearpage

% Title page
\thispagestyle{empty}

\begin{center}
\bf\large
Final Report:\\
\Huge
Iron Legacy
\end{center}

\vspace{4em}

\begin{center}
\textbf{Team 9}\\
Jonathan Farr\\
Philip Garrison\\
Greg Kownacki\\
Billy Wood
\end{center}

\vspace{4em}

\begin{center}
\textbf{Submitted to}\\
Thomas M. Keating\\
Assistant Teaching Professor\\
School of Computer Science\\
Carnegie Mellon University
\end{center}

\vspace{4em}

\begin{center}
\textbf{Prepared by}\\
\begin{minipage}[c][3.7em][t]{.45\textwidth}
\begin{center}
Billy Wood\\
Mellon College of Science\\
Carnegie Mellon University
\end{center}
\end{minipage}
\begin{minipage}[c][3.7em][t]{.45\textwidth}
\begin{center}
Philip Garrison\\
Carnegie Institute of Technology\\
Carnegie Mellon University
\end{center}
\end{minipage}\\
May 7, 2014
\end{center}

\vfill

% TODO: formatting

\begin{abstract}
This report describes \emph{Iron Legacy}, a turn-based strategy game that aims 
to innovate on traditional strategy game models. We created the game with 
Python and Pygame. We met weekly and used Git to collaborate. We failed to 
complete the game; most notably, it is lacking a computer AI. Because of this, 
the game is neither fun nor challenging, although it is still playable by two
people taking turns on a single computer. Due to the same time constraints that 
prevented us from completing the game, there has been no outside testing of 
\emph{Iron Legacy}. We learned valuable lessons on team organization and 
software management, which we hope future groups can learn from.
% Total: 112 words, according to google docs and wordcounter.net
\end{abstract}

\clearpage

% Table of Contents
\pagenumbering{roman}
\tableofcontents
% TODO
% Make sure sub/section names match those in the
% \addcontentsline command

\clearpage

% Introduction
\pagenumbering{arabic}
\addcontentsline{toc}{section}{Introduction}
\section*{Introduction}
This section first introduces the problem our project aims to solve and some
games similar to ours.
Then we comment on the value of the literature review and take a look at our
solution and its level of success.

\addcontentsline{toc}{subsection}{Problem}
\subsection*{Problem}
Games are a fun diversion from work, although many games focus on only one type
of strategic thinking. Our team recognized that there is a need for a game with
greater diversity of scale which can enable a richer, more challenging,
strategic experience.

\addcontentsline{toc}{subsection}{Background}
\subsection*{Background}
We designed \emph{Iron Legacy} with the games 
\emph{Fire Emblem}\cite{FireEmblem} and \emph{Civilization}\cite{Civilization} 
very much in mind.
Broadly, \emph{Fire Emblem} focuses on tactics in individual battles, while
\emph{Civilization} focuses on overarching war strategy.
Our project, \emph{Iron Legacy}, is intended to bridge this gap.

\addcontentsline{toc}{subsection}{Literature Review}
\subsection*{Literature Review}
Over the course of the project, we looked to three resources in particular:
``Towards an Evenly Match Opponent AI in Turn-based Strategy Games,''\cite{AI} 
a paper on adaptive-strength AI, 
``A Survey of Procedural Terrain Generation Techniques using Evolutionary 
Algorithms,''\cite{Terrain Generation} a paper that helped us dynamically
generate the world map, and a forum post ``The Pixel Art 
Tutorial''\cite{Pixel Art} for creating images in the game.
In general, these were very useful.
We developed our terrain generation algorithm using an algorithm from the 
second paper, and we designed our art using the tips from the pixel art
tutorial.
The AI paper was only of limited utility, because we did not create an AI for
the game, although it did inform our thoughts on a potential AI design.

\addcontentsline{toc}{subsection}{Solution}
\subsection*{Solution}
Our solution is to create a strategy game in which the player will fight on
both a larger world map and several smaller local instances.
We intended to create a fun and challenging gameplay experience by developing
an intuitive interface and a challenging computer AI.

\addcontentsline{toc}{subsection}{Success}
\subsection*{Success}
\emph{Iron Legacy} does not succeed in addressing the problem.
We did not create the AI as intended, and without an AI, meaningful 
single-player gameplay is impossible.
This is a serious issue, but the game is otherwise playable.
We will elaborate on the difficulties that led to this result later in the 
report.

% Approach
\addcontentsline{toc}{section}{Approach}
\section*{Approach}
The objective of this project was to build a turn-based strategy game with 
shifting-scale battles.
Our four main goals were to make the game fun, challenging, strategic, and 
simple to play from a user interface perspective.

From a technical standpoint, we needed a framework in which we could write the
game logic, handle user input, and update the display in real time.
We considered the Python\cite{Python}, Java\cite{Java}, and 
Javascript\cite{Javascript} languages for writing the game. 
We settled on Python since we all knew Python already and we had some ideas of
game libraries we could use.
The two libraries we looked at were Pygame\cite{Pygame} and 
TkInter\cite{TkInter}.
Through Jonathan's past experience using TkInter in 15-112 and Greg's 
experience with Pygame, we chose to eschew TkInter in favor of Pygame.

For collaboration and version control, our team used Git \cite{Git} to
synchronize documents and code.

% Results
\addcontentsline{toc}{section}{Results}
\section*{Results}

Our project, \emph{Iron Legacy}, was unsuccessful in accomplishing
    our goals.
    The final product allows squads to move around on a macro-scale world-map.
    When squads collide, they spawn a new micro-scale battle.
    This was the intended core of the game, but the final product is lacking
    key features.

The most notable missing feature is the AI.
    The game must be played by multiple players on the same computer.
    This breaks one of the desired aspects of the game,
    namely that the world-map would run in real-time.
    Since players are on the same computer, they must take turns when making
    moves, even on the macro-scale.

We had also intended for the macro-scale to be able to influence the
    micro-scale.
    However, due to time constraints, we were unable to add this feature.
    Thus, the micro-battles and the world-map are mostly independent,
    which removes a key element of strategy that we had intended.

In our proposal, we had stated that our goal was to make a fun, challenging,
    strategic, and simple game.
    However, without an AI, it is hard to guage the ``challenge'' of our game.
    As mentioned, the strategic elements were lessened by the lack of
    interaction between the micro-battles and the world-map.
    Our game also lacks certain graphical elements, such as which squares
    you can attark and how much health each unit has, making the interface
    confusing.

Our proposal gave very vague measures for evaluation.
    In particular, it said that ``the game must be free of all bugs that
    inhibit the natural flow of the game.''
    From there, we suggested that we would make the game publicly available,
    once it was playable.
    Last, we suggested that there should be no single-best tactic in the game,
    but that all tactics should have advantages and disadvantages.

As the game only recently reached a playable state,
    we never had the oportunity to make it publicly available.
    As such, we never received outside opinions about the quality of the game.
    Moreover, we cannot say whether any strategies are ``best,''
    as we have not collected any data from games.

Last, while the game is now playable, it is certainly not ``free of all bugs.''
    There are a few issues with the square highlighting that shows
    where a unit can move.
    Further, the game occasionally crashes when a unit is killed.
    This blatantly inhibits the natural flow of the game.

% Discussion
\addcontentsline{toc}{section}{Discussion}
\section*{Discussion}

In this section, we will discuss the lessons we learned from working 
    on a large group project.
    We will then make recommendations for how to improve \emph{Iron Legacy}
    or create other games of a similar nature.

\addcontentsline{toc}{subsection}{Lessons Learned}
\subsection*{Lessons Learned}

Our biggest lesson was the importance of meeting as a group.
    When we began discussing \emph{Iron Legacy}, the entire team was thrilled
    by the idea.
    However, we all prioritized other assignments above this game,
    and so we did not make much progress during the first month of development.
    Because we didn't have much progress, we didn't think it was worth
    getting together as a group.
    In our experience, when we did meet, we worked very productively
    and enthusiastically.
    Had we begun by scheduling a weekly meeting time,
    we probably would have stayed closer to our original Gantt chart.

We also learned the importance of designing your interfaces before you start
    programming.
    As a team, we wrote code in a very \emph{ad hoc} manner.
    For instance, in writing the interface for the player to control the game,
    we put all of the logic for unit movement into the key-listener's
    event-handler.
    While this makes sense for the player, it means that the AI would have
    to press a key in order to do anything.
    This was one of the primary reasons that we never wrote AI.
    In reality, we needed to move all of the logic to separate functions that
    both the AI and the key-listener could call.
    As time went on, the task of refactoring our code to the interface model
    became increasingly intimidating, and it was never accomplished.

Finally, we learned a lot about reading other people's code.
    This was the first time that any of us had worked on a team project,
    and it was unusual for us to read anyone's code except for our own.
    This taught us the importance of including comments that describe what
    ambiguous variables are for, and to make sure our code is generally
    readable.

\addcontentsline{toc}{subsection}{Recommendations}
\subsection*{Recommendations for Improvement and Future Groups}

The next step for \emph{Iron Legacy} would be AI.
    As mentioned in the lessons learned, this means refactoring a large
    chunk of the code such that there is a single interface
    of functions for the AI to call in order to control units.

After that, we would suggest making the game available in a beta-test,
    to see which features players like and dislike.
    We recommend that the tema gives out a survey with quantitative
    responses in order to get hard-data on which features most need
    improvement.

During testing, we suggest that the team gather data on which squads
    win and lose during the micro-battles.
    This would provide quantitative data that would show if one unit
    is stronger than the others.
    From here, the next team could weigh the strengths and
    weaknesses of the different play-strategies, in order to make
    the game as balanced as possible.

% TODO: Include Gantt charts, maybe other pictures

% Sources Cited
\addcontentsline{toc}{section}{References}
\begin{thebibliography}{99}

\bibitem{FireEmblem}
\emph{Fire Emblem} - {\tt http://fireemblem.nintendo.com/}

\bibitem{Civilization}
\emph{Civilization} - {\tt http://civilizaton.com}

% TODO format these better
\bibitem{AI}
Potisartra, K., \& Kotrajaras, V. Towards an Evenly Match AI in Turn-based Strategy Games. Accessed May 5, 2014. http://www.cp.eng.chula.ac.th/~vishnu/gameProg/papers/ CGAT\_TowardsEvenlyMatchTBS-FINAL.pdf.

\bibitem{Terrain Generation}
Raffe, W., Zambetta, F., \& Li, X. A Survey of Procedural Terrain Generation Techniques using Evolutionary Algorithms, June 10, 2013. Accessed May 5, 2014. http://goanna.cs.rmit.edu.au/~xiaodong/publications/ptg-raffe-cec2012.PDF.

\bibitem{Pixel Art}
Cure. Cure to Pixel Joint web forum, "The Pixel Art Tutorial," November 27, 2010. Accessed April 8, 2014. http://www.pixeljoint.com/forum/forum\_posts.asp?TID=11299. 


\bibitem{Python}
Python - {\tt http://www.python.org}

\bibitem{Java}
Java - {\tt http://www.java.com}

\bibitem{Javascript}
Javascript - {\tt http://developer.mozilla.org/en-US/docs/Web/JavaScript}

\bibitem{Pygame}
Pygame - {\tt http://www.pygame.org}

\bibitem{TkInter}
TkInter - {\tt http://wiki.python.org/moin/TkInter}

\bibitem{Git}
Git - {\tt http://www.github.com}

\end{thebibliography}

\end{document}
